\documentclass[a4paper,11pt]{jsarticle}

% 数式
\usepackage{amsmath,amsfonts}
\usepackage{bm}
% 画像
\usepackage[dvipdfmx]{graphicx}
\usepackage{wrapfig}
% bibTeX
\usepackage[backend = biber,style =apa, sorting =none,]{biblatex}
% listings(ソースコード)
\usepackage{listings}

\lstset{
  basicstyle={\ttfamily},
  identifierstyle={\small},
  commentstyle={\smallitshape},
  keywordstyle={\small\bfseries},
  ndkeywordstyle={\small},
  stringstyle={\small\ttfamily},
  frame={tb},
  breaklines=true,
  columns=[l]{fullflexible},
  numbers=left,
  xrightmargin=0zw,
  xleftmargin=3zw,
  numberstyle={\scriptsize},
  stepnumber=1,
  numbersep=1zw,
  lineskip=-0.5ex
}
\renewcommand{\lstlistingname}{}

% 参考文献ファイルのリスト
\addbibresource{references.bib}

% 等式番号を章ごとに割り振る
\makeatletter
\@addtoreset{equation}{section}
\def\theequation{\thesection.\arabic{equation}}
\makeatother

\begin{document}

\title{計算機科学実験2ソフトウェア報告書3}
\author{安済翔真}
\date{\today}
\maketitle

Task2で用いたエージェント(以下Task2Agent)を改良してMainTask3に取り組んだ。

\section*{実施内容}

まず、すでに実装されているTask2Agentを用いてMainTask3を実行した。
Task2Agentには敵を避けるコードが書かれていないため、敵に何度もぶつかって死んでしまった。

\begin{figure}[h]
  \centering
  \includegraphics*[width=50mm]{
    images/report3/task2-agent.png
  }
  \caption[図1]{}
\end{figure}

そこで、まずは常にファイアボールを打ちながら走っていくというエージェントを試した。
しかし、このエージェントではファイアボールを発射するタイミングが合わず敵を倒しきれないことがあり、途中までは
上手く進んだものの程なくして敵にぶつかってしまった。一度ファイアマリオの状態からスーパマリオの状態に
なってしまうと、敵をファイアボールで倒すことができないため、すぐに死んでしまった。また、常にファイアボールを
打ち続けると、ダッシュボタンを押し続けることができないため、マリオの進むスピードが遅くなってしまっていた。


\begin{wrapfigure}{r}{50mm}
  \begin{center}
    \includegraphics*[width=50mm]{images/report3/falling-enemy.png}
    \caption[図2]{}
  \end{center}
\end{wrapfigure}

これに対処するため、マリオの目の前(1マス前)に敵がいるとき、マリオの状態がファイアマリオの場合はファイアボールを打ち、
そうでない場合はジャンプをする、というエージェントを試した。実行してみると、ダッシュ状態では1マス前の敵に対して対応する
のでは対処が遅く、そのまま敵にぶつかってしまうことが何度かあったため、2マス前に敵がいる場合もファイアボールを打ったり
ジャンプをしたりという処理を行うようにした。しかし、この方法でも、上から降ってくる敵に対処できず、
敵が大量に現れる場面で死んでしまった。

そこで最後に、マリオの右上に落下中の敵がいる場合は、一定時間左に進んで敵を退避した後、再度進み出すという
エージェントを作成した。その結果、見事ステージをクリアすることができた。

\section*{実行結果}

\begin{lstlisting}[caption=実行結果]
  [~ Mario AI Benchmark ~ 0.1.9]

  [MarioAI] ~ Evaluation Results for Task: BasicTask
          Evaluation lasted : 31615 ms
          Weighted Fitness : 8304
              Mario Status : WIN!
                Mario Mode : small
  Collisions with creatures : 2
      Passed (Cells, Phys) : 256 of 256, 4096 of 4096 (100% passed)
  Time Spent(marioseconds) : 51
    Time Left(marioseconds) : 148
              Coins Gained : 14 of 23 (60% collected)
        Hidden Blocks Found : 0 of 0 (0% found)
        Mushrooms Devoured : 0 of 0 found (0% collected)
          Flowers Devoured : 0 of 0 found (0% collected)
                kills Total : 36 of 102 found (35%)
              kills By Fire : 21
            kills By Shell : 0
            kills By Stomp : 15
      PunJ : 0

  min = 21.0
  max = 21.0
  ave = 21.0
  sd  = NaN
  n   = 1
\end{lstlisting}

\begin{wrapfigure}{r}{52mm}
  \begin{center}
    \includegraphics*[width=50mm]{images/report3/reaching-goal.png}
  \end{center}
\end{wrapfigure}

\section*{結論と考察}
今回作成したエージェントでは、偶然敵の間を潜り抜けてクリアできたという印象があり、真に敵を避けることが
できるエージェントではなかった。実際、このエージェントでは敵を適切に避けることができない場合が
多々あり、MainTask3をクリアした際も何度かダメージを受けた状態でゴールしていた。敵に対してもっと適切に
対処できるエージェントを作る必要があると感じた。

\printbibliography[title=参考文献]

\end{document}
